%! Author = Robert Kulesa, Daniel Liu, Michael Li
%! Date = 10/5/2021

% Preamble
\documentclass[11pt]{article}

% Packages
\usepackage{amsmath}

% Document
\begin{document}
    \begin{titlepage}
        \begin{center}
            \vspace*{1cm}

            \Huge
            \textbf{Fast Trajectory Replanning}

            \vspace{0.5cm}
            \LARGE
            Assignment 1

            \vspace{1cm}

            \textbf{Robert Kulesa, Michael Li, Daniel Liu}

            \vfill


            \vspace{0.8cm}

            \Large
            CS440 Fall 2021\\
            Professor Boularias\\
            Rutgers University - New Brunswick\\
            October 15, 2021

        \end{center}
    \end{titlepage}
    \begin{center}
        \Large
        \textbf{Part 1 - Understanding the methods}
    \end{center}
    \normalsize
    \begin{enumerate}
        \item[a)] The agent moves east, because the unblocked,
        unvisited neighbor with the lowest cost $f(x) = g(x) + h(x)$ is the eastern neighbor.
        Using manhattan distance as $h(x)$, the eastern neighbor has $f(x) = 1 + 2 = 3$,
        whereas the northern neighbor has $f(x) = 1 + 4 = 5$.
        Therefore, the eastern neighbor is selected, and the agent explores the eastern cell.
        \item[b)] second item
    \end{enumerate}
    
    
    \begin{center}
        \Large
        \textbf{Part 2 - The Effects of Ties}
    \end{center}
    %insert explanation here
    
    \begin{center}
        \Large
        \textbf{Part 3 - Forward vs. Backward}
    \end{center}
    %insert explanation here
    
    \begin{center}
        \Large
        \textbf{Part 4 - Heuristics in the Adaptive A*}
    \end{center}
    %insert explanation here
    
    
    \begin{center}
        \Large
        \textbf{Part 5 - Repeated Forward A* vs. Adaptive A*}
    \end{center}
    %insert explanation here
    
    
    \begin{center}
        \Large
        \textbf{Part 6 - Statistical Significance}
    \end{center}
    %insert explanation here
    
    
\end{document}
