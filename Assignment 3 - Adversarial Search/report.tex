%! Author = Robert Kulesa, Daniel Liu, Michael Li
%! Date = 11/4/2021

% Preamble
\documentclass[11pt]{article}

% Packages
\usepackage{amsmath}
\usepackage{graphicx}

% Document
\begin{document}
    \begin{titlepage}
        \begin{center}
            \vspace{1cm}

            \Huge
            \textbf{Adversarial Search - Bayesian Networks}

            \vspace{0.5cm}
            \LARGE
            Assignment 3

            \vspace{1cm}

            \textbf{Michael Li - 192008938}

            \textbf{Daniel Liu - 184007283}

            \textbf{Robert Kulesa - 185009892}


            \vfill


            \vspace{0.8cm}

            \Large
            CS440 Fall 2021\\
            Professor Boularias\\
            Rutgers University - New Brunswick\\
            December 7, 2021

        \end{center}
    \end{titlepage}

    \begin{center}
        \Large
        \textbf{Problem 1}
    \end{center}
    \normalsize
    \includegraphics[width=\linewidth]{images/game_tree} \\
    "?" values are used to annotate loop states.
    When an agent has to choose between a +1 value and a "?" value (+1,?)
    the max player will always choose +1 while the min player will choose ?,
    and vice versa for (-1,?), with the min player choosing -1 and the max player choosing "?".
    Since the game value of a loop state is unknown, it is best for a max/min player to choose
    the corresponding max/min value if it is available, however if that value is not
    available the "?" state is a viable option to explore as it cannot be worse than an immediate loss.
    If all the successors of a state have a "?" value, the backed-up value is also "?".

    \begin{center}
        \Large
        \textbf{Problem 2}
    \end{center}
    \normalsize
    \begin{enumerate}
        \item[(a)] Computing the joint probability distribution:
            \begin{gather*}
                P(a, b, c, d, e) = P(a)*P(b)*P(c)*P(d|a,b)*P(e|b,c)\\
                P(a, b, c, d, e) = 0.2*0.5*0.8*0.1*0.3\\
                P(a, b, c, d, e) = 0.0024
            \end{gather*}
        \item[(b)] Computing the joint probability distribution:
            \begin{gather*}
                P(\neg{a}, \neg{b}, \neg{c}, \neg{d}, \neg{e}) = P(\neg{a})*P(\neg{b})*P(\neg{c})*P(\neg{d}|\neg{a},\neg{b})*P(\neg{e}|\neg{b},\neg{c})\\
                P(\neg{a}, \neg{b}, \neg{c}, \neg{d}, \neg{e}) = 0.8*0.5*0.2*0.1*0.8\\
                P(\neg{a}, \neg{b}, \neg{c}, \neg{d}, \neg{e}) = 0.0064
            \end{gather*}
        \item[(c)] Using the conditional probability rule $P(A|B) = \frac{P(A\bigcap B)}{P(B)}$
            \begin{gather*}
                P(\neg{a} | b, c, d, e) = \frac{P(\neg{a}, b, c, d, e)}{P(b, c, d, e)}\\
                P(\neg{a} | b, c, d, e) = \frac{P(\neg{a}, b, c, d, e)}{P(a, b, c, d, e) + P(\neg{a}, b, c, d, e)}\\
                P(\neg{a} | b, c, d, e) = \frac{P(\neg{a})*P(b)*P(c)*P(d|\neg{a},b)*P(e|b,c)}{0.0024 + P(\neg{a})*P(b)*P(c)*P(d|\neg{a},b)*P(e|b,c)}\\
                P(\neg{a} | b, c, d, e) = \frac{0.8*0.5*0.8*0.6*0.3}{0.0024 + 0.8*0.5*0.8*0.6*0.3}\\
                P(\neg{a} | b, c, d, e) = \frac{0.0576}{0.0024 + 0.0576}\\
                P(\neg{a} | b, c, d, e) = 0.96
            \end{gather*}
    \end{enumerate}

    \begin{center}
        \Large
        \textbf{Problem 3}
    \end{center}
    \normalsize
    \begin{enumerate}
        \item[(a)]
        \item[(b)]
    \end{enumerate}

    \begin{center}
        \Large
        \textbf{Problem 4}
    \end{center}
    \normalsize
    \begin{enumerate}
        \item[(a)]
        \item[(b)]
    \end{enumerate}

\end{document}
