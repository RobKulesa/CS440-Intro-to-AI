%! Author = Robert Kulesa, Daniel Liu, Michael Li
%! Date = 11/4/2021

% Preamble
\documentclass[11pt]{article}

% Packages
\usepackage{amsmath}

% Document
\begin{document}
    \begin{titlepage}
        \begin{center}
            \vspace{1cm}

            \Huge
            \textbf{Search Problems in AI}

            \vspace{0.5cm}
            \LARGE
            Assignment 2

            \vspace{1cm}

            \textbf{Michael Li - 192008938}

            \textbf{Daniel Liu - 184007283}

            \textbf{Robert Kulesa - 185009892}


            \vfill


            \vspace{0.8cm}

            \Large
            CS440 Fall 2021\\
            Professor Boularias\\
            Rutgers University - New Brunswick\\
            November 12, 2021

        \end{center}
    \end{titlepage}

    \begin{center}
        \Large
        \textbf{Problem 1}
    \end{center}
    \normalsize
    The sequence of nodes expanded by A* search, given the tree,
    and straight line distance heuristics, is in the following order:
    \begin{enumerate}
        \item[0.] $\left(city, f(city), g(city), h(city)\right)$
        \item $\left(\text{Lugoj}, 244, 0, 244\right)$
        \item $\left(\text{Mehadia}, 311, 70, 241\right)$
        \item $\left(\text{Lugoj}, 384, 140, 244\right)$
        \item $\left(\text{Drobeta}, 387, 145, 242\right)$
        \item $\left(\text{Craiova}, 425, 265, 160\right)$
        \item $\left(\text{Timisoara}, 440, 111, 329\right)$
        \item $\left(\text{Lugoj}, 446, 222, 244\right)$
        \item $\left(\text{Mehadia}, 451, 210, 241\right)$
        \item $\left(\text{Mehadia}, 461, 220, 241\right)$
        \item $\left(\text{Pitesti}, 503, 403, 100\right)$
        \item $\left(\text{Bucharest}, 504, 504, 0\right)$
    \end{enumerate}

    \begin{center}
        \Large
        \textbf{Problem 2}
    \end{center}
    \normalsize
    \begin{enumerate}
        \item[(a)] %insert explanation here
        \item[(b)] %insert explanation here
    \end{enumerate}

    \begin{center}
        \Large
        \textbf{Problem 3}
    \end{center}
    \normalsize
    \begin{enumerate}
        \item[(a)] %insert explanation here
        \item[(b)] %insert explanation here
        \item[(c)] %insert explanation here
        \item[(d)] %insert explanation here
        \item[(e)] %insert explanation here
        \item[(f)] %insert explanation here
        \item[(g)] %insert explanation here
        \item[(h)] %insert explanation here
        \item[(i)] %insert explanation here
    \end{enumerate}

    \begin{center}
        \Large
        \textbf{Problem 4}
    \end{center}
    \normalsize
    %insert explanation here

    \begin{center}
        \Large
        \textbf{Problem 5}
    \end{center}
    \normalsize
    %insert explanation here

    \begin{center}
        \Large
        \textbf{Problem 6}
    \end{center}
    \normalsize
    %insert explanation here

    \begin{center}
        \Large
        \textbf{Problem 7}
    \end{center}
    \normalsize
    \begin{enumerate}
        \item[(a)] %insert explanation here
        \item[(b)] %insert explanation here
        \item[(c)] %insert explanation here
    \end{enumerate}

    \begin{center}
        \Large
        \textbf{Problem 8}
    \end{center}
    \normalsize
    \begin{enumerate}
        \item[(a)] %insert explanation here
        \item[(b)] %insert explanation here
        \item[(c)] %insert explanation here
    \end{enumerate}

    \begin{center}
        \Large
        \textbf{Problem 9}
    \end{center}
    \normalsize
    \begin{enumerate}
        \item[(a)] %insert explanation here
        \item[(b)] %insert explanation here
        \item[(c)] %insert explanation here
        \item[(d)] %insert explanation here
        \item[(e)] %insert explanation here
        \item[(f)] %insert explanation here
    \end{enumerate}


\end{document}
